\section{Results}
To explore the utility of the tool, I applied the pipeline to generate valid IVFs corresponding to 7 different
traits or biomarkers and estimated the causal effect score with myocardial infarction (MI) data 
from CARdioGrAM (22,233 cases and 64,762 controls), and type 2 diabetes (T2D) data combining DIAGRAM stage 1 (12,171 
cases and 56,862 controls) and stage 2 (22,669 cases and 58,119 controls), both from publicly available meta-analysis consortia (CARdioGrAM and DIAGRAM cite)

First, I analyzed trait-disease associations for which there is considerable evidence for association. For MI, higher LDL-C and SBP are widely accepted to each be causal for increasing risk of heart disease and MI incidents, while for HDL-C, both epidemiological and previously reported MR work has found no causal relationship with MI (Voight cite). For T2D, while there is no there are no clear causal traits for use as a control, higher fasting glucose levels is a standardized bio-marker to test for pre-diabetes and was used as a positive control for T2D. Results from the pipeline presented in Figure 1 show that our positive controls for MI (LDL-C and 
SBP) and for T2D (FastGlu), produce a result consistent with previous epidemiological observation and MR reports. 

Our primary analysis makes the assumption that weights used for our score are unbiased, and one well-known source of potential bias is winner's curse, a phenomenon where positive discoveries exceeding a p-value threshold and that are underpowered to identify effect size from the sample size will estimate effect sizes that are biased upward (Xiao 2011). To validate the robustness of our observations, I examined which, if any, of our selected SNPs may be subject to this effect. Upon careful inspection of our control IVFs, I found that our IVFs for HDL-C and LDL-C contained SNPs predominately from Willer et al. [188,577 samples], the most recent and comprehensive association study on lipid levels, with some SNP data originating from older studies. 

In our LDL-C IVFs for T2D and MI, two and three SNPs, respectively, were from separate, older studies: Katherisan et al. from 2008 [~30,000 samples], Aulchenko et al. from 2008 [~20,000 samples], Kim et al. from 2011 [??? see 21909109], and Coram et al.  from June 2013 [?http://www.sciencedirect.com/science/article/pii/S0002929713002127]. Each SNP had a proxy in the Willer et al. primary data and was referenced to update effect values. For the LDL-T2D IVF, rs12713956 (mapping to APOB) from Coram et al had a reported effect of .147 (originally 4.86 mg/dl and converted to s.d units by 33mg/dl/sd conversion factor) and was updated to 0.0719 s.d, and  rs6756629 (mapping to ABCG5) from Aulchenko et al. was updated from 0.16 to 0.13 s.d. effect. The LDL-MI IVF additionally contained rs6102059 (mapping to MAFB) from Kathiresan et al. with a reported effect of .06 s.d. and updated to 0.03 s.d. Applying these updates and rerunning our analysis with T2D and MI resulted in consistent results with our original IVFs, with the new LDL-MI result (OR = 1.44) closer to that which is observed in epidemiology (OR=1.54). We repeated this sanity check for our HDL instrument which also contained several SNPs with effects from pre-Willer et al. analysis. For HDL IVFs, rs16940212 (mapping to LIPC ??) from Kim et al. was updated from 0.02 mg/dl effect (likely misreported unit?) to 0.0464 s.d. effect, and rs7395662 (mapping to MADD/FOLH1from Aulchenko et al. was updated from 0.07 s.d to 0.0196 s.d. effect. In addition to these two SNPs, in the HDL-MI IVF had rs12979813 (mapping to LOC55908) from Coram et al. was updated from 0.02 to 0.0321 s.d. effect(allele directions seem off!!,haven't actually updated yet, necessary?). The resulting updated analysis yielded consistent non-association results with both T2D and MI, although it is notable that the HDL-MI result (OR=0.91,p=0.36) moved closer to that of previous MR reports (OR=0.93,p=0.63). 

Having passed preliminary control tests for robustness, I applied the pipeline to several examples to generate hypotheses and produced results of potential interest. Specifically, for SBP and Bone Mineral Density, powered by 14 and 47 SNPs respectively, our pipeline estimated p $<$ 0.005 and p $<$ 0.02 significant positive causal effects with T2D. Systolic blood pressure is one of the traits included in metabolic syndrome, known to increase risk of heart disease and diabetes. However, no direct relationship has been established between blood pressure and diabetes as a potential causal factor. In addition, there have been few studies on the associations of bone mineral density and diabetes, but with no well-established connection. These preliminary findings prompt further refinement of the IVFs and more in-depth analyses to confirm and explore these results.

\section{Results}
To explore the utility of the tool, I applied the pipeline to generate valid IVFs corresponding to 7 different
traits or biomarkers and estimated the causal effect score with myocardial infarction (MI) data 
from CARdioGrAM (22,233 cases and 64,762 controls), and type 2 diabetes (T2D) data combining DIAGRAM stage 1 (12,171 
cases and 56,862 controls) and stage 2 (22,669 cases and 58,119 controls), both from publicly available meta-analysis consortia (CARdioGrAM and DIAGRAM cite)

First, I analyzed trait-disease associations for which there is considerable evidence for association. For MI, higher LDL-C and SBP are widely accepted to each be causal for increasing risk of heart disease and MI incidents, while for HDL-C, both epidemiological and previously reported MR work has found no causal relationship with MI (Voight cite). For T2D, while there is no there are no clear causal traits for use as a control, higher fasting glucose levels is a standardized bio-marker to test for pre-diabetes and was used as a positive control for T2D. Results from the pipeline presented in Figure 1 show that our positive controls for MI (LDL-C and 
SBP) and for T2D (FastGlu), produce a result consistent with previous epidemiological observation and MR reports. 

Upon careful inspection of our control IVFs, I noticed that for HDL-C and LDL-C, although most SNPs came from the recent Willer et al., there were a few that came from older lipid studies. When updating these SNPs from more recent non-reported Willer et al. data (excluded from NHGRI catalog), we see the HDL-MI result move closer to non-association result expected from previous reports (see supplementary materials).

Next, I applied the pipeline to several examples to generate hypotheses and produced results of potential interest. Specifically, for SBP and Bone Mineral Density, powered by 14 and 47 SNPs respectively, our pipeline estimated p $<$ 0.005 and p $<$ 0.02 significant positive causal effects with T2D. Systolic blood pressure is one of the traits included in metabolic syndrome, known to increase risk of heart disease and diabetes. However, no direct relationship has been established between blood pressure and diabetes as a potential causal factor. In addition, there have been few studies on the associations of bone mineral density and diabetes, but with no well-established connection. These preliminary findings prompt further refinement of the IVFs and more in-depth analyses to confirm and explore these results.